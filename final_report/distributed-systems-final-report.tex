\documentclass{scrartcl}
\usepackage[utf8]{inputenc}
\usepackage{hyperref}
\usepackage{url}
\usepackage{natbib}
\usepackage{graphicx}
\usepackage{cleveref} % this must be the last package to be loaded

\newcommand{\emailaddr}[1]{\href{mailto:#1}{\texttt{#1}}}

\title{\LARGE
    Final Report
}

% Consider watching:
% https://www.youtube.com/watch?v=ihxSUsJB_14
% https://www.youtube.com/watch?v=XTFWaV55uDo

\author{
    Nicolò Monaldini \\ \emailaddr{nicolo.monaldini@studio.unibo.it}
}

\date{February 2025}

\begin{document}

\maketitle

\begin{abstract}
    This project aims to produce a survey on the topic of serverless computing titled ``An Overview on Serverless Computing, its Opportunities and its Challenges''. The survey's abstract is reported below: 
    
    Serverless computing has gained importance in both academia and industry due to its cost-efficiency, automated scalability, and virtually unlimited resource availability, offering a novel pay-as-you-go model compared to traditional cloud computing solutions. The serverless paradigm not only eases end-users of the burden of infrastructure management, but its cost model enables users to pay only for resources consumed, while the provider bears the cost of idle resources. However, the stateless nature of the services offered and communication needs of related executions are problems that need to be addressed, as they could pose performance issues and limit the applications of the paradigm. This survey provides an overview of the current advancements in serverless computing, examining its opportunities, challenges, and potential solutions. 
\end{abstract}

\newpage
\section{Concept}\label{concept}

The project intends to realize a survey to provide an overview on serverless computing, specifically about its opportunities and its challenges.

\subsection{Covered topics}

In order to provide a comprehensive study on serverless computing, this survey analyzes the relationship between serverless computing and other areas, such as security or caching. It will cover the following topics:

\begin{itemize}
  \item The serverless architecture
  \item How serverless differs from traditional cloud computing paradigms
  \item The cost model of serverless and its advantages
  \item Security and Privacy
  \begin{itemize}
    \item Authentication and authorization in serverless applications
    \item Security concerns in the context of serverless
  \end{itemize}
  \item Implementation of caching mechanisms in serverless applications
  \item Performance insights of serverless applications and architecture modifications aimed to increase performance of serverless platforms
  \item Main scheduling approaches for serverless functions
  \item Statistics about the usage of serverless platforms
\end{itemize}

\newpage
\section{Implementation}
The survey is divided in the following sections:

\begin{itemize}
  \item Introduction
  \item Cost-effectiveness 
  \item Security and Privacy 
  \item Data Caching 
  \item Performance 
  \item Scheduling 
  \item Statistics on Serverless Usage 
  \item Conclusion
\end{itemize}

As it is further explained in the introduction, to produce the study various articles and surveys were initially analyzed to provide general information about the serverless paradigm. The improvements that serverless brings and its critical issues are discussed, and then studies that attempt to tackle them are reviewed in detail.

\subsection{Introduction}
The introduction:
\begin{itemize}
  \item Explains what Function-as-a-Service and Backend-as-a-Service are
  \item Explains the functioning of the architecture employed by serverless platforms, describing the parts involved
  \item Introduces the differences from other traditional cloud computing paradigms
  \item Explains the methodology used to produce the survey
\end{itemize}

\subsection{Cost-effectiveness}
This section explains the pay-as-you-go cost model employed by serverless platforms, how it differs from the cost model used by Infrastructure-as-a-Service platforms, and what its advantages are, for both providers and clients.

\subsection{Security and Privacy}
This section describes how authentication and authorization are implemented in serverless architectures. It also highlights the need for privacy describing scenarios where a lack of proper security could result in data leakages.

\subsection{Data Caching}
This section highlights the incompatibilities that the serverless architecture has with typical caching mechanisms and what the alternatives to provide caching are. An open source implementation of caching in serverless environments is closely described.

\subsection{Performance}
This section analyzes the limitations that High-Performance Computing applications, and, more in general, latency sensitive applications encounter when using a serverless approach. An alternative serverless architecture is described, focusing on the novel concepts it introduces to limit execution delays.

\subsection{Scheduling}
Scheduling policies for functions executions that allow to avoid resource contention and delays are described, in particular in connection with the following topics:
\begin{itemize}
  \item Resource Consumption Patterns
  \item Reducing Startup Overhead
  \item Data Locality
\end{itemize}

\subsection{Statistics on Serverless Usage}
General statistics about the usage of serverless platforms are provided. This section analyzes a report from Datadog\footnote{\url{https://www.datadoghq.com/}} and a quantitative study about the AWS Serverless Application Marketplace, a marketplace by Amazon offering functions ready to be deployed.

\subsection{Conclusion}
The conclusion summarizes the contribution of the survey. It acknowledges the strengths of the serverless approaches, as well as the challenges posed by the stateless nature and latency issues of serverless computing, which need to be addressed for broader adoption.

\newpage
\section{Self-evaluation}
As explained in the previous sections, the work both provides a general overview on serverless computing and in-depth descriptions of some studies that aim to solve specific issues. I would say that this is the main strength of the survey: it provides both general and detailed information about the discussed topics. On the other hand, I would say that its main weakness lays in the fact that it does not match the scope of surveys published in research venues in terms of number of publications reviewed. This work reviews, among others, surveys that provide insights based on multiple publications; however, most published surveys, often the result of collaborative efforts, review hundreds of publications, resulting in a much broader scope.

\end{document}
